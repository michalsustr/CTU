%% Vloženie rôznych nastavení
\documentclass[pdftex,12pt,a4paper]{article}

%% Zoznam použitých balíčkov
\usepackage[slovak]{babel}
\usepackage[utf8]{inputenc}
\usepackage{fancyhdr}
\usepackage{mathtools}
\usepackage{hyperref}
\hypersetup{pdfborder={0 0 0}}
\usepackage[final]{pdfpages}
\usepackage{graphicx}
\usepackage{epstopdf}
\usepackage{subfigure}
\usepackage[top=3cm, bottom=3cm, right=2.5cm, left=2.5cm]{geometry}
\usepackage{listings}
\usepackage{color}
\usepackage{textcomp}

%% Nastavenie zvýrazňovača syntaxe vo výpise kódov
\definecolor{listinggray}{gray}{0.9}
\definecolor{lbcolor}{rgb}{0.95,0.95,0.95}
\lstset{
	backgroundcolor=\color{lbcolor},
	tabsize=4,
	rulecolor=,
	language=matlab,
        basicstyle=\scriptsize,
        upquote=true,
        aboveskip={0.3\baselineskip},
        columns=fixed,
        showstringspaces=false,
        extendedchars=true,
        breaklines=true,
        prebreak = \raisebox{0ex}[0ex][0ex]{\ensuremath{\hookleftarrow}},
        frame=single,
        showtabs=false,
        showspaces=false,
        showstringspaces=false,
        identifierstyle=\ttfamily,
        keywordstyle=\color[rgb]{0,0,1},
        commentstyle=\color[rgb]{0.133,0.545,0.133},
        stringstyle=\color[rgb]{0.627,0.126,0.941},
}

%% Stĺpce matíc
\makeatletter
\renewcommand*\env@matrix[1][*\c@MaxMatrixCols c]{%
  \hskip -\arraycolsep
  \let\@ifnextchar\new@ifnextchar
  \array{#1}}
\makeatother

%% Vlastné príkazy
% norm - norma vektora ||x||
\newcommand{\norm}[1]{\left|\left|#1\right|\right|}
% adj - adjungovaná matica 
\newcommand{\adj}{\operatorname{adj}}
% rank - hodnosť matice
\newcommand{\rank}{\operatorname{rank}}
% m - skratka na výpis tučného písma v rovniciach (pre matice)
\newcommand{\m}[1]{\mathbf{#1}} 

\usepackage{tikz}

\setlength{\parskip}{10pt plus 1pt minus 1pt}
\setlength{\parindent}{0in}

\def \nazovUlohy{Lab 3 - Improving the flawed server}
\def \skola{\textbf{Chalmers university - Computer science}}
\def \tema{Homework from computer networks}
\def \author{\textbf{Michal Šustr}, \textbf{Arnaud D'Artigues}}
\def \email{\href{mailto:michal.sustr@gmail.com}{\texttt{michal.sustr@gmail.com}} \\ \href{mailto:arnaud.dartigues@gmail.com}{\texttt{arnaud.dartigues@gmail.com}}}
\def \date{13. 10. 2014}

%% Pokračovanie nastavení

\fancyhead[L]{\skola \\ \nazovUlohy} 
\fancyhead[R]{\author \\}
\fancyfoot[C]{\thepage}
\renewcommand{\headrulewidth}{0.4pt}
\renewcommand{\footrulewidth}{0.4pt}
\renewcommand{\refname}{\section{Zoznam použitej literatúry}}

\hypersetup{
    bookmarks=true,         % show bookmarks bar?
    unicode=false,          % non-Latin characters in Acrobat’s bookmarks
    pdftoolbar=true,        % show Acrobat’s toolbar?
    pdfmenubar=true,        % show Acrobat’s menu?
    pdffitwindow=false,     % window fit to page when opened
    pdfstartview={FitH},    % fits the width of the page to the window
    pdftitle=\nazovUlohy,    % title
    pdfauthor=\author,     % author
    pdfsubject=\tema,   % subject of the document
    pdfnewwindow=true,      % links in new window
    colorlinks=true,       % false: boxed links; true: colored links
    linkcolor=blue,          % color of internal links
    citecolor=green,        % color of links to bibliography
    filecolor=magenta,      % color of file links
    urlcolor=blue           % color of external links
}

\begin{document}

%% Úvodná stránka

%% Hlavička na prvej strane dokumentu
\begin{center}
	\makebox[4cm][l]{\includegraphics[width=3cm]{lev.png}}
	\parbox[s][4cm][s]{10cm}{\scshape \large
	Chalmers university\\Computer science department}
	\vskip 2cm {\Huge \bfseries \nazovUlohy}
	\vskip 1cm {\Large \tema}
\end{center}

\vfill

%% Päta
\begin{flushright}
	\author \\
	\email \\ 
	~\\
	\date
\end{flushright}

\newpage
\pagestyle{fancy}


\section{Answers to lab questions}
\subsection*{Question I.a.1}
The response to messages from 2 clients connected at the same time are immediate. Both clients have the the state ESTABLISHED in netstat:
\begin{verbatim}
tcp        0      0 localhost:53286         localhost:5703          ESTABLISHED
tcp        0      0 localhost:5703          localhost:53284         ESTABLISHED
tcp        0      0 localhost:5703          localhost:53286         ESTABLISHED
tcp        0      0 localhost:53284         localhost:5703          ESTABLISHED
\end{verbatim}

\subsection*{Question I.a.2}
The result of the simulation is 
\begin{verbatim}
Simulating 100 clients.
Establishing 100 connections... 
  successfully initiated 100 connection attempts!
Connect timing results for 100 successful connections
  - min time: 0.798271 ms
  - max time: 4.469441 ms
  - average time: 2.405017 ms
 (0 connections failed!)
Roundtrip timing results for 100 connections for 10000 round trips
  - min time: 3269.909818 ms
  - max time: 3971.980467 ms
  - average time: 3803.638460 ms
\end{verbatim}

Just 7 connections with 255 round trips took on average $40639ms$ last time, which is a significant performance improvement.

\subsection*{Question I.a.3}
We managed to create DOS attack by modifying the client-multi file, there are too many connections for the server to handle. However, we couldn't do it by just one simple client connecting to the server, even if it requested the server many times.


%\begin{thebibligraphy}{5}
%\textsc{	\bibitem{ari}
%		\emph{Polynomiální metody - Michael Šebek} \\
%		\url{http://www.polyx.com/_ari/slajdy/Pr-ARI-19-Polynom.pdf}
%}
%\end{thebibliography}

\end{document}